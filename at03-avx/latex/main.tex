\documentclass[conference]{IEEEtran}
\usepackage[brazilian]{babel}
\usepackage[utf8]{inputenc}
\usepackage[T1]{fontenc}
\usepackage{cite}
\ifCLASSINFOpdf
  \usepackage[pdftex]{graphicx}
  % declare the path(s) where your graphic files are
  % \graphicspath{{../pdf/}{../jpeg/}}
  % and their extensions so you won't have to specify these with
  % every instance of \includegraphics
  % \DeclareGraphicsExtensions{.pdf,.jpeg,.png}
\else
  % or other class option (dvipsone, dvipdf, if not using dvips). graphicx
  % will default to the driver specified in the system graphics.cfg if no
  % driver is specified.
  \usepackage[dvips]{graphicx}
  % declare the path(s) where your graphic files are
  % \graphicspath{{../eps/}}
  % and their extensions so you won't have to specify these with
  % every instance of \includegraphics
  % \DeclareGraphicsExtensions{.eps}
\fi
\usepackage[cmex10]{amsmath}
\usepackage{algorithmic}
\usepackage{array}
\usepackage{mdwmath}
\usepackage{mdwtab}
%\usepackage{eqparbox}
\usepackage[tight,footnotesize]{subfigure}
%\usepackage[caption=false]{caption}
\usepackage[font=footnotesize]{subfig}
%\usepackage{stfloats}
\usepackage{url}
\usepackage{csvsimple,longtable,booktabs}
\usepackage{pgfplots}
\usepackage{pgfplotstable}
\hyphenation{op-tical net-works semi-conduc-tor}
\usepackage{listings}

\lstset{language=C++,
        basicstyle=\ttfamily,
        keywordstyle=\color{blue}\ttfamily,
        stringstyle=\color{red}\ttfamily,
        commentstyle=\color{green}\ttfamily,
        morecomment=[l][\color{magenta}]{\#}
}


\ifCLASSINFOpdf
  \usepackage[pdftex]{graphicx}
  % declare the path(s) where your graphic files are
  % \graphicspath{{../pdf/}{../jpeg/}}
  % and their extensions so you won't have to specify these with
  % every instance of \includegraphics
  % \DeclareGraphicsExtensions{.pdf,.jpeg,.png}
  \else
  % or other class option (dvipsone, dvipdf, if not using dvips). graphicx
  % will default to the driver specified in the system graphics.cfg if no
  % driver is specified.
  \usepackage[dvips]{graphicx}
  % declare the path(s) where your graphic files are
  % \graphicspath{{../eps/}}
  % and their extensions so you won't have to specify these with
  % every instance of \includegraphics
  % \DeclareGraphicsExtensions{.eps}
\fi
\usepackage[cmex10]{amsmath}
\usepackage{algorithmic}
\usepackage{array}
\usepackage{mdwmath}
\usepackage{mdwtab}
%\usepackage{eqparbox}
\usepackage[tight,footnotesize]{subfigure}
%\usepackage[caption=false]{caption}
\usepackage[font=footnotesize]{subfig}
%\usepackage{stfloats}
\usepackage{url}
\usepackage{csvsimple,longtable,booktabs}
\usepackage{pgfplots}
\usepackage{pgfplotstable}
\hyphenation{op-tical net-works semi-conduc-tor}

\begin{document}
\title{Programação de Alto Desempenho\\
\large Atividade 3 - Otimização por vetorização}

\author{\IEEEauthorblockN{Lucas Santana Lellis - 69618}
\IEEEauthorblockA{PPGCC - Instituto de Ciência e Tecnologia\\
	Universidade Federal de São Paulo} }

% make the title area
\maketitle

%\IEEEpeerreviewmaketitle

\section{Introdução}
Nesta atividade foram realizados experimentos relacionados com a vetorização de laços, que pode ser realizada de forma automática pelo computador, mas também utilizando-se funções intrínsecas das arquiteturas sse, avx e avx2.
Cada experimento foi realizado 5 vezes, e os resultados apresentados são a média dos resultados obtidos em cada um deles.

Todos os programas foram feitos em C, com as flags ``-O3 -msse -ftree-vectorize -fopt-info-vec-all'', utilizando a biblioteca PAPI para estimar o tempo total de processamento, o total de operações de ponto flutuante (PAPI\_SP\_OPS), e o fator de Ciclos por elementos do vetor (CPE).

As especificações da máquina utilizada estão disponíveis na Tabela \ref{tab:cpu}.

\begin{table}[htb!]
\centering
\caption{Especificações da Máquina}
\label{tab:cpu}
\begin{tabular}{rl}
\bf{PAPI Version            } & 5.4.3.0 \\
\bf{Model string and code   } & Intel Core i5-2400\\
\bf{CPU Revision            } & 7.000000 \\
\bf{CPU Max Megahertz       } & 3101 \\
\bf{CPU Min Megahertz       } & 1600 \\
\bf{Threads per core        } & 1 \\
\bf{Cores per Socket        } & 4 \\
\bf{Number Hardware Counters} & 11 \\
\bf{Max Multiplex Counters  } & 192 \\
\bf{Cache L3                } & 6144 KB \\
\bf{Cache L2                } & 256 KB * 4 \\
\bf{RAM                     } & 4 Gb \\
\bf{SO                      } & Ubuntu 14.04 x64 \\
\bf{Kernel                  } & 3.13.0-46-generic \\
\bf{GCC                     } & 6.1.1 20160511\\
\end{tabular}
\end{table}

\section{Experimento 1}
Nesse experimento foram feitos diversos testes envolvendo 6 diferentes tipos de cálculos, sendo avaliada a capacidade do compilador em vetorizá-los automaticamente, sendo feitas também as adaptações possíveis e necessárias para que isso seja possível, sendo também implementadas versões vetorizadas fazendo-se uso de funções intrínsecas do padrão avx. Em todos os experimentos, foram utilizados vetores de tamanho $50000000$, iniciados de forma pseudo-aleatoria, com seed fixa de valor $424242$.

\subsection{Algoritmo A}
O primeiro algoritmo consiste no seguinte cálculo:

\begin{lstlisting}
  for (i=1; i<N; i++) {
    x[i] = y[i] + z[i];
    a[i] = x[i-1] + 1.0;
  }
\end{lstlisting}

Ao compilar este exemplo com a flag ``-fopt-info-vec-all'', pudemos observar a seguinte mensagem: ``src/exercicio\_a.c:25:3: note: LOOP VECTORIZED'', que coincide exatamente com o laço descrito acima.

Assim, foi feita uma segunda versão, agora vetorizada por funções instrínsecas desse mesmo algoritmo:

\begin{lstlisting}
  ones = _mm256_set1_ps(1.0);
  for( i = 1; i < 8; i++ ) {
    x[ i ] = y[ i ] + z[ i ];
    a[ i ] = x[ i - 1 ] + 1.0;
  }
  for( i = 8; i < N - 8; i += 8 ) {
    v1 = _mm256_load_ps( y + i );
    v2 = _mm256_load_ps( z + i );
    v3 = _mm256_add_ps( v1, v2 );
    _mm256_store_ps( x + i, v3 );

    v1 = _mm256_load_ps( x + i -1 );
    v2 = _mm256_add_ps( v1, ones );
    _mm256_store_ps( a + i, v2 );
  }
  for( ; i < N; i++ ) {
    x[ i ] = y[ i ] + z[ i ];
    a[ i ] = x[ i - 1 ] + 1.0;
  }
\end{lstlisting}

A corretude do algoritmo é facilmente observável pelas operações aritméticas realizadas, e também pelos resultados obtidos ao final da execução dos dois programas.
Em um teste padronizado, em que os vetores são inicializados com os mesmos valores pseudo-aleatórios, temos o seguinte resultado nas dez ultimas posições do vetor $a$: %\cite{wiki:BLAS}

\begin{table}[htb!]
\centering
\begin{tabular}{ l c c c c c c c c c c }
  Versao original & 173.41 & 195.57 & 75.57 & 156.35 & 283.08 \\
  & 22.80 & 357.32 & 338.45 & 254.81 & 265.32\\
  Versao vetorizada & 173.41 & 195.57 & 75.57 & 156.35 & 283.08 \\
  & 22.80 & 357.32 & 338.45 & 254.81 & 265.32\\
\end{tabular}
\end{table}

A comparação do desempenho se dá então pela Tabela \ref{tab:exp1.01}.

\begin{table}[htb!]
	\centering
	\caption{Análise de desempenho do experimento 1 a}
	\label{tab:exp1.01}
	\begin{tabular}{lrrrr}%
		\bfseries Mode & \bfseries Tempo($\mu{s}$)& \bfseries L2\_DCM & \bfseries MFLOPS & \bfseries CPE
		\csvreader[]{tables/ex_a.csv}{}
		{\\\csvcoli & \csvcolii & \csvcoliii & \csvcoliv & \csvcolv}
	\end{tabular}
\end{table}

\subsection{Algoritmo B}
O algoritmo consiste no seguinte cálculo:

\begin{lstlisting}
  for (i=0; i<N-1; i++) {
    x[ i ] = y[ i ] + z[ i ];
    a[ i ] = x[ i+1 ] + 1.0;
  }
\end{lstlisting}

Ao compilar este exemplo com a flag ``-fopt-info-vec-all'' não pudemos observar o texto: ``note: LOOP VECTORIZED'', e por isso, foi feita uma pequena alteração para habilitar a vetorização automática desse laço.

\begin{lstlisting}
  for( i = 0; i < N - 1; i++ ) {
    a[ i ] = x[ i + 1 ] + 1.0;
    x[ i ] = y[ i ] + z[ i ];
  }
\end{lstlisting}

Ao compilar este exemplo com a flag ``-fopt-info-vec-all'', pudemos observar a seguinte mensagem: ``src/exercicio\_b.c:25:3: note: LOOP VECTORIZED''.

Assim, foi feita uma terceira versão, agora vetorizada por funções instrínsecas desse mesmo algoritmo:

\begin{lstlisting}
  ones = _mm256_set1_ps( 1.0 );
  for( i = 0; i < N - 9; i += 8 ) {
    v1 = _mm256_load_ps( x + i + 1 );
    v2 = _mm256_add_ps( v1, ones );
    _mm256_store_ps( a + i, v2 );
    v1 = _mm256_load_ps( y + i );
    v2 = _mm256_load_ps( z + i );
    v2 = _mm256_add_ps( v1, v2 );
    _mm256_store_ps( x + i, v2 );
  }
  for( ; i < N - 1; i++ ) {
    a[ i ] = x[ i + 1 ] + 1.0;
    x[ i ] = y[ i ] + z[ i ];
  }
\end{lstlisting}

A corretude do algoritmo é garantida pois a inversão da relação de antidependência não altera o resultado, e em contrapartida permite que o algoritmo seja paralelizado automaticamente pelo compilador. A mesma lógica foi utilizada na versão com paralelização intrínseca. Em testes padronizados, em que os vetores são inicializados com os mesmos valores pseudo-aleatórios, temos na Tabela  nas dez ultimas posições do vetor $a$, e em seguida o vetor $x$: %\cite{wiki:BLAS}


\begin{table}[htb!]
\centering
\begin{tabular}{l c c c c c c c c c c }
  Versao original & 203.17 & 111.63 & 76.05 & 171.88 & 48.31 \\
  & 73.78 & 19.93 & 169.58 & 129.03 & 23.91 \\
  Versao vetorizada & 203.17 & 111.63 & 76.05 & 171.88 & 48.31 \\
  & 73.78 & 19.93 & 169.58 & 129.03 & 23.91 \\
  Versao intrinseca & 203.17 & 111.63 & 76.05 & 171.88 & 48.31 \\
  & 73.78 & 19.93 & 169.58 & 129.03 & 23.91
\end{tabular}
\label{tab:exp2.testea}
\end{table}


\begin{table}[htb!]
\centering
\begin{tabular}{ l c c c c c c c c c c }
  Versao original & 194.57 & 74.57 & 155.35 & 282.08 & 21.80\\
  & 356.32 & 337.45 & 253.81 & 264.32 & 128.03 \\
  Versao vetorizada & 194.57 & 74.57 & 155.35 & 282.08 & 21.80\\
  & 356.32 & 337.45 & 253.81 & 264.32 & 128.03 \\
  Versão Intrinseca & 194.57 & 74.57 & 155.35 & 282.08 & 21.80\\
  & 356.32 & 337.45 & 253.81 & 264.32 & 128.03
\end{tabular}
\label{tab:exp2.testeb}
\end{table}

A comparação do desempenho se dá então pela Tabela \ref{tab:exp1.02}

\begin{table}[htb!]
	\centering
	\caption{Análise de desempenho do experimento 1 b}
	\label{tab:exp1.02}
	\begin{tabular}{lrrrr}%
		\bfseries Mode & \bfseries Tempo($\mu{s}$)& \bfseries L2\_DCM & \bfseries MFLOPS & \bfseries CPE
		\csvreader[]{tables/ex_b.csv}{}
		{\\\csvcoli & \csvcolii & \csvcoliii & \csvcoliv & \csvcolv}
	\end{tabular}
\end{table}


\subsection{Algoritmo C}
O algoritmo consiste no seguinte cálculo:

\begin{lstlisting}
  for( i = 0; i < N - 1; i++ ) {
    x[ i ] = a[ i ] + z[ i ];
    a[ i ] = x[ i + 1 ] + 1.0;
  }
\end{lstlisting}

Ao compilar este exemplo com a flag ``-fopt-info-vec-all'' não pudemos observar o texto: ``note: LOOP VECTORIZED'', e por isso, foi feita uma pequena alteração para habilitar a vetorização automática desse laço, neste caso, a falsa relação de dependência foi corrigida a partir da utilização de uma variável auxiliar.

\begin{lstlisting}
  for( i = 0; i < N - 1; i++ ) {
    aux = a[ i ] + z[ i ];
    a[ i ] = x[ i + 1 ] + 1.0;
    x[ i ] = aux;
  }
\end{lstlisting}

Ao compilar este exemplo com a flag ``-fopt-info-vec-all'', pudemos observar a seguinte mensagem: ``src/exercicio\_c.c:22:3: note: LOOP VECTORIZED''.

Assim, foi feita uma terceira versão, agora vetorizada por funções instrínsecas desse mesmo algoritmo:

\begin{lstlisting}
  ones = _mm256_set1_ps( 1.0 );
  for( i = 0; i < N - 9; i += 8 ) {
    v1 = _mm256_load_ps( a + i );
    v2 = _mm256_load_ps( z + i );
    v3 = _mm256_add_ps( v1, v2 ); //aux

    v1 = _mm256_load_ps( x + i + 1);
    v1 = _mm256_add_ps( v1, ones);
    _mm256_store_ps( a + i, v1 );
    _mm256_store_ps( x + i, v3 );
  }
  for( ; i < N - 1; i++ ) {
    x[ i ] = a[ i ] + z[ i ];
    a[ i ] = x[ i + 1 ] + 1.0;
  }
\end{lstlisting}

A corretude do algoritmo é garantida e não está sujeita à variações nos resultados.
Em testes padronizados, em que os vetores são inicializados com os mesmos valores pseudo-aleatórios, temos na Tabela  nas dez ultimas posições do vetor $a$, e em seguida o vetor $x$: %\cite{wiki:BLAS}

\begin{table}[htb!]
\centering
\begin{tabular}{l c c c c c c c c c c }
  Versao original & 208.92 & 209.19 & 197.92 & 79.58 & 107.30\\
   & 189.40 & 54.15 & 71.08 & 49.35 & 99.84 \\
  Versao vetorizada & 208.92 & 209.19 & 197.92 & 79.58 & 107.30\\
   & 189.40 & 54.15 & 71.08 & 49.35 & 99.84 \\
  Versao intrinseca & 208.92 & 209.19 & 197.92 & 79.58 & 107.30\\
   & 189.40 & 54.15 & 71.08 & 49.35 & 99.84
\end{tabular}
\label{tab:exp2.testea}
\end{table}


\begin{table}[htb!]
\centering
\begin{tabular}{ l c c c c c c c c c c }
  Versao original & 48.50 & 101.70 & 75.22 & 307.19 & 370.98\\
   & 259.06 & 263.70 & 259.64 & 158.99 & 48.35 \\
  Versao vetorizada & 48.50 & 101.70 & 75.22 & 307.19 & 370.98\\
   & 259.06 & 263.70 & 259.64 & 158.99 & 48.35 \\
  Versão Intrinseca & 48.50 & 101.70 & 75.22 & 307.19 & 370.98\\
   & 259.06 & 263.70 & 259.64 & 158.99 & 48.35
\end{tabular}
\label{tab:exp2.testeb}
\end{table}

A comparação do desempenho se dá então pela Tabela \ref{tab:exp1.03}

\begin{table}[htb!]
	\centering
	\caption{Análise de desempenho do experimento 1 c}
	\label{tab:exp1.03}
	\begin{tabular}{lrrrr}%
		\bfseries Mode & \bfseries Tempo($\mu{s}$)& \bfseries L2\_DCM & \bfseries MFLOPS & \bfseries CPE
		\csvreader[]{tables/ex_c.csv}{}
		{\\\csvcoli & \csvcolii & \csvcoliii & \csvcoliv & \csvcolv}
	\end{tabular}
\end{table}


\subsection{Algoritmo D}
O primeiro algoritmo consiste no seguinte cálculo:
\begin{lstlisting}
  for( i = 0; i < N; i++ ) {
    t = y[ i ] + z[ i ];
    a[ i ] = t + 1.0 / t;
  }
\end{lstlisting}

Ao compilar este exemplo com a flag ``-fopt-info-vec-all'', pudemos observar a seguinte mensagem: ``src/exercicio\_d.c:25:3: note: LOOP VECTORIZED'', que coincide exatamente com o laço descrito acima.

Assim, foi feita uma segunda versão, agora vetorizada por funções instrínsecas desse mesmo algoritmo:

\begin{lstlisting}
  ones = _mm256_set1_ps( 1.0 );
  for( i = 0; i < N - 8; i += 8 ) {
    v1 = _mm256_load_ps( y + i );
    v2 = _mm256_load_ps( z + i );
    v1 = _mm256_add_ps( v1, v2 );
    v2 = _mm256_div_ps( ones, v1 );
    v1 = _mm256_add_ps( v1, v2 );
    _mm256_store_ps( a + i, v1 );
  }
  for( ; i < N; i++ ) {
    t = y[ i ] + z[ i ];
    a[ i ] = t + 1.0 / t;
  }
\end{lstlisting}

A corretude do algoritmo é facilmente observável pelas operações aritméticas realizadas, e também pelos resultados obtidos ao final da execução dos dois programas.
Em um teste padronizado, em que os vetores são inicializados com os mesmos valores pseudo-aleatórios, temos o seguinte resultado nas dez ultimas posições do vetor $a$:

%\cite{wiki:BLAS}
\begin{table}[htb!]
\centering
\begin{tabular}{ l c c c c c c c c c c }
  Versao original & 194.58 & 74.58 & 155.35 & 282.08 & 21.85 \\
  & 356.33 & 337.45 & 253.82 & 264.33 & 286.58 \\
  Versao vetorizada & 194.58 & 74.58 & 155.35 & 282.08 & 21.85 \\
  & 356.33 & 337.45 & 253.82 & 264.33 & 286.58 \\
\end{tabular}
\end{table}


A comparação do desempenho se dá então pela Tabela \ref{tab:exp1.04}.

\begin{table}[htb!]
	\centering
	\caption{Análise de desempenho do experimento 1 d}
	\label{tab:exp1.04}
	\begin{tabular}{lrrrr}%
		\bfseries Mode & \bfseries Tempo($\mu{s}$)& \bfseries L2\_DCM & \bfseries MFLOPS & \bfseries CPE
		\csvreader[]{tables/ex_d.csv}{}
		{\\\csvcoli & \csvcolii & \csvcoliii & \csvcoliv & \csvcolv}
	\end{tabular}
\end{table}

\subsection{Algoritmo e}
O primeiro algoritmo consiste no seguinte cálculo:
\begin{lstlisting}
  for( i = 0; i < N; i++ ) {
    s += z[ i ];
  }
\end{lstlisting}

Embora seja um exemplo simples, não pudemos observar a vetorização automática de uma operação de redução, e a única opção seria habilitar outras funções do compilador como a flag ``fast-math'', que resultou numa variação muito grande dos resultados, e por esse motivo não foi adotada.

Assim, foi feita uma segunda versão, agora vetorizada por funções instrínsecas desse mesmo algoritmo, onde os dados são armazenados em um acumulador, e somados ao final.

\begin{lstlisting}
  acc = _mm256_xor_ps( acc, acc );
  for( i = 0; i < N - 8; i += 8 ) {
    data = _mm256_load_ps( z + i );
    acc = _mm256_add_ps( acc, data );
  }
  for( ; i < N; i++ ) {
    s += z[ i ];
  }
  for( i = 0; i < 8; i++ ) {
    s += acc[ i ];
  }
\end{lstlisting}

A corretude do algoritmo é facilmente observável pelas operações aritméticas realizadas, mas é esperada uma variação nos resultados pela mudança na ordem em que são realizadas as operações de ponto flutuante.
Em um teste padronizado, em que os vetores são inicializados com os mesmos valores pseudo-aleatórios, temos no primeiro algoritmo o valor de $s = 535541824.0$ e no segundo, o valor de $ s = 535541856.0 $. Trata-se de uma diferença grande mas ainda aceitável se comparada à outras alternativas.

A comparação do desempenho se dá então pela Tabela \ref{tab:exp1.05}.

\begin{table}[htb!]
	\centering
	\caption{Análise de desempenho do experimento 1 e}
	\label{tab:exp1.05}
	\begin{tabular}{lrrrr}%
		\bfseries Mode & \bfseries Tempo($\mu{s}$)& \bfseries L2\_DCM & \bfseries MFLOPS & \bfseries CPE
		\csvreader[]{tables/ex_e.csv}{}
		{\\\csvcoli & \csvcolii & \csvcoliii & \csvcoliv & \csvcolv}
	\end{tabular}
\end{table}


\subsection{Algoritmo f}
O primeiro algoritmo consiste no seguinte cálculo:
\begin{lstlisting}
  for( i = 1; i < N; i++ ) {
    a[ i ] = a[ i - 1 ] + b[ i ];
  }
\end{lstlisting}

Diferente dos exercícios anteriores, este algoritmo possui uma dependência real e intratável. A análise do desempenho desse algoritmo se dá então pela Tabela \ref{tab:exp1.06}.

\begin{table}[htb!]
	\centering
	\caption{Análise de desempenho do experimento 1 f}
	\label{tab:exp1.06}
	\begin{tabular}{lrrrr}%
		\bfseries Mode & \bfseries Tempo($\mu{s}$)& \bfseries L2\_DCM & \bfseries MFLOPS & \bfseries CPE
		\csvreader[]{tables/ex_f.csv}{}
		{\\\csvcoli & \csvcolii & \csvcoliii & \csvcoliv & \csvcolv}
	\end{tabular}
\end{table}


% \begin{figure}[htb!]
% \centering
% \caption{Comparação do tempo de execução entre as diferentes hierarquias.}
% \begin{tikzpicture}
% \begin{axis}[xlabel={Tamanho}, ylabel={Tempo ($\mu{s}$)}, legend pos=north west]
% \addplot gnuplot [raw gnuplot] {plot 'plots/ex01-time.txt' index 0};
% \addplot gnuplot [raw gnuplot] {plot 'plots/ex01-time.txt' index 1};
% \addplot gnuplot [raw gnuplot] {plot 'plots/ex01-time.txt' index 2};
% \addplot gnuplot [raw gnuplot] {plot 'plots/ex01-time.txt' index 3};
% \addplot gnuplot [raw gnuplot] {plot 'plots/ex01-time.txt' index 4};
% \addplot gnuplot [raw gnuplot] {plot 'plots/ex01-time.txt' index 5};
% \legend{$IJK$, $IKJ$, $JIK$, $JKI$, $KIJ$, $KJI$ }
% \end{axis}
% \end{tikzpicture}
% \label{fig:exp01-time}
% \end{figure}


% \begin{figure}[htb!]
% 	\centering
% 	\begin{tikzpicture}
% 	\begin{axis}[xlabel={Tamanho}, ylabel={Cache misses}, legend pos=north west]
% 	\addplot gnuplot [raw gnuplot] {plot 'plots/ex01-l2dcm.txt' index 0};
% 	\addplot gnuplot [raw gnuplot] {plot 'plots/ex01-l2dcm.txt' index 1};
% 	\addplot gnuplot [raw gnuplot] {plot 'plots/ex01-l2dcm.txt' index 2};
% 	\addplot gnuplot [raw gnuplot] {plot 'plots/ex01-l2dcm.txt' index 3};
% 	\addplot gnuplot [raw gnuplot] {plot 'plots/ex01-l2dcm.txt' index 4};
% 	\addplot gnuplot [raw gnuplot] {plot 'plots/ex01-l2dcm.txt' index 5};
% 	\legend{$IJK$, $IKJ$, $JIK$, $JKI$, $KIJ$, $KJI$ }
% 	\end{axis}
% 	\end{tikzpicture}
% 	\caption{Comparação de cache-misses em memória cache L2 entre as diferentes hierarquias.}
% 	\label{fig:exp01-l2dcm}
% \end{figure}

%
% \section{Conclusão}
% Os experimentos realizados demonstraram a vantagem computacional obtida por meio da utilização de técnicas que favorecem a afinidade de memória, adequando-se a hierarquia dos laços, e também o impacto causado por diferentes estruturas de dados, de forma que se faça o uso mais eficiente do processador, evitando que este permaneça ocioso.
%
% Porém, isso não é o suficiente para se obter um desempenho ótimo do algoritmo, uma vez que os resultados variam de acordo com a complexidade dos problemas, e uma vez que desconhecemos todos os recursos adicionais presentes na arquitetura do processador, que o BLAS certamente faz uso. Isso também evidencia a superioridade e a importância da utilização de bibliotecas otimizadas de operações de álgebra linear na computação científica.
%
%
% Também notamos que a utilização de fusão de laços não necessariamente trará benefícios relevantes, pois o desempenho depende da natureza dos dados e também das operações realizadas à cada iteração. Já a utilização de diferentes estruturas de dados poderia apresentar diferenças mais expressivas em casos mais complexos, uma vez que não notamos uma diferença relevante nos experimentos simples realizados.
%
% Finalmente, a utilização das flags de otimização pode prejudicar a avaliação da diferença real do desempenho causada por cache misses, uma vez que recursos do processador, como as operações vetoriais, possam ser ativados automaticamente.

% \section{Experimento XXXX}
% Nesse experimento foi utilizada a função ..... do BLAS
%
% \begin{table}[htb!]
% 	\centering
% 	\caption{Desempenho obtido no exp XXXX}
% 	\label{tab:exp02}
% 	\begin{tabular}{ccccc}%
% 		\bfseries Size & \bfseries Tempo($\mu{s}$)& \bfseries L2\_DCM & \bfseries MFLOPS & \bfseries CPI
% 		\csvreader[]{tables/ex0XXXX.csv}{}
% 		{\\\csvcoli & \csvcolii & \csvcoliii & \csvcoliv & \csvcolv & \csvcolvi}
%
% 	\end{tabular}
% \end{table}
%
%
%
% \begin{figure}[htb!]
% 	\centering
% 	\begin{tikzpicture}
% 	\begin{axis}[xlabel={Tamanho}, ylabel={Tempo ($\mu{s}$)}, legend pos=north west]
% 	\addplot gnuplot [raw gnuplot] {plot 'plots/ex0XXXX-time.txt' index 0};
% 	\addplot gnuplot [raw gnuplot] {plot 'plots/ex0XXXX-time.txt' index 1};
% 	\addplot gnuplot [raw gnuplot] {plot 'plots/ex0XXXX-time.txt' index 2};
% 	\legend{$16$, $32$ , $64$ }
% 	\end{axis}
% 	\end{tikzpicture}
% 	\caption{Comparação do tempo de execução entre os diferentes tamanhos minimos.}
% 	\label{fig:exp0XXXX-time}
% \end{figure}
%
%
% \begin{figure}[htb!]
% 	\centering
% 	\begin{tikzpicture}
% 	\begin{axis}[xlabel={Tamanho}, ylabel={Cache misses}, legend pos=north west]
% 	\addplot gnuplot [raw gnuplot] {plot 'plots/ex0XXXX-l2dcm.txt' index 0};
% 	\addplot gnuplot [raw gnuplot] {plot 'plots/ex0XXXX-l2dcm.txt' index 1};
% 	\addplot gnuplot [raw gnuplot] {plot 'plots/ex0XXXX-l2dcm.txt' index 2};
% 	\legend{$16$, $32$ , $64$ }
% 	\end{axis}
% 	\end{tikzpicture}
% 	\caption{Comparação de cache-misses em memória cache L2 entre os diferentes tamanhos minimos.}
% 	\label{fig:exp0XXXX-l2dcm}
% \end{figure}



% \bibliographystyle{IEEEtran}
% \bibliography{IEEEabrv,references}
\end{document}
