\documentclass[conference]{IEEEtran}
\usepackage[brazilian]{babel}
\usepackage[utf8]{inputenc}
\usepackage[T1]{fontenc}
\usepackage{cite}
\ifCLASSINFOpdf
  \usepackage[pdftex]{graphicx}
  % declare the path(s) where your graphic files are
  % \graphicspath{{../pdf/}{../jpeg/}}
  % and their extensions so you won't have to specify these with
  % every instance of \includegraphics
  % \DeclareGraphicsExtensions{.pdf,.jpeg,.png}
\else
  % or other class option (dvipsone, dvipdf, if not using dvips). graphicx
  % will default to the driver specified in the system graphics.cfg if no
  % driver is specified.
  \usepackage[dvips]{graphicx}
  % declare the path(s) where your graphic files are
  % \graphicspath{{../eps/}}
  % and their extensions so you won't have to specify these with
  % every instance of \includegraphics
  % \DeclareGraphicsExtensions{.eps}
\fi
\usepackage[cmex10]{amsmath}
\usepackage{algorithmic}
\usepackage{array}
\usepackage{mdwmath}
\usepackage{mdwtab}
%\usepackage{eqparbox}
\usepackage[tight,footnotesize]{subfigure}
%\usepackage[caption=false]{caption}
\usepackage[font=footnotesize]{subfig}
%\usepackage{stfloats}
\usepackage{url}
\usepackage{csvsimple,longtable,booktabs}
\usepackage{pgfplots}
\usepackage{pgfplotstable}
\hyphenation{op-tical net-works semi-conduc-tor}
\usepackage{listings}

\lstset{language=C++,
	basicstyle=\ttfamily,
	keywordstyle=\color{blue}\ttfamily,
	stringstyle=\color{red}\ttfamily,
	commentstyle=\color{green}\ttfamily,
	morecomment=[l][\color{magenta}]{\#}
}

\begin{document}
\title{Programação de Alto Desempenho\\
\large Atividade 4 - Otimizações independentes de compilador ou maquina}

\author{\IEEEauthorblockN{Lucas Santana Lellis - 69618}
\IEEEauthorblockA{PPGCC - Instituto de Ciência e Tecnologia\\
	Universidade Federal de São Paulo} }

% make the title area
\maketitle

%\IEEEpeerreviewmaketitle

\section{Introdução}
Nesta atividade foram realizados experimentos relacionados com a otimização de código independente de compilador ou máquina, e também na utilização de recursos básicos do gprof.
Cada experimento foi realizado 5 vezes, e os resultados apresentados são a média dos resultados obtidos em cada um deles.

Todos os programas foram feitos em C, com otimização -O3, utilizando a biblioteca PAPI para estimar o tempo total de processamento, e o número de operações por ciclo.


As especificações da máquina utilizada estão disponíveis na Tabela \ref{tab:cpu}.

\begin{table}[htb!]
\centering
\caption{Especificações da Máquina}
\label{tab:cpu}
\begin{tabular}{lr}
 CPU & Intel Core i5 - 3470\\
 Cores & 4\\
 Threads & 4\\
 Clock & 3.2 GHz\\
 Cache L3 & 6144 KB \\
 Cache L2 & 256 KB * 4 \\
 Hardware Counters & 11 \\
 RAM & 8 Gb \\
 SO & Fedora 23 \\
 Kernel & 4.7.4 \\
 GCC & 5.3.1\\

\end{tabular}
\end{table}

\section{Experimento I - Otimização do código do Jogo da Vida }

Neste experimento foi realizada a comparação da alteração de desempenho causada pela redução do número de multiplicações no endereçamento dos elementos da matriz utilizada para representar o jogo da vida.

Assim, no primeiro programa, chamado de ``Padrão'', o cálculo de cada vizinho de uma coordenada $(i,j)$ envolve multiplicações e somas, como o código abaixo:

\begin{lstlisting}
up = val[(i-1)*n + j];
upright = val[(i-1)*n + j+1];
right = val[i*n + j+1];
rightdown = val[(i+1)*n + j+1];
down = val[(i+1)*n + j];
downleft = val[(i+1)*n + j-1];
left = val[i*n + j-1];
leftup = val[(i-1)*n + j-1];
\end{lstlisting}

Enquanto o segundo programa, chamado de ``Otimizado'', o cálculo da posição atual é dado por $inj = i * n + j$, e os vizinhos são obtidos pelo deslocamento em função da soma e subtração de valores conhecidos:

\begin{lstlisting}
int inj = i*n + j;
up = val[inj - n];
upright = val[inj - n + 1];
right = val[inj + 1];
rightdown = val[inj + n + 1];
down = val[inj + n];
downleft = val[inj + n - 1];
left = val[inj - 1];
leftup = val[inj - n - 1];
\end{lstlisting}

Tendo como configuração inicial um tabuleiro de tamanho 10x10 com células vivas nas posições (1,2), (2,3), (3,1), (3,2), (3,3) e as demais células mortas. Nas Figuras de \ref{fig:life1-10-start} até \ref{fig:life1-10-end} a configuração do tabuleiro nas quatro primeiras iterações do algoritmo.
\begin{figure}[!htb]
\begin{verbatim}
+------------+
|            |
|  0         |
|   0        |
| 000        |
|            |
|            |
|            |
|            |
|            |
|            |
|            |
|            |
+------------+
\end{verbatim}
\caption{Condições iniciais do teste em tabuleiro 10x10.}
\label{fig:life1-10-start}
\end{figure}

\begin{figure}[!htb]
	\begin{verbatim}
+------------+
|            |
|            |
| 0 0        |
|  00        |
|  0         |
|            |
|            |
|            |
|            |
|            |
|            |
|            |
+------------+
	\end{verbatim}
	\caption{Condições do tabuleiro 10x10 após 1a iteração.}
\end{figure}
\begin{figure}[!htb]
	\begin{verbatim}
+------------+
|            |
|            |
|   0        |
| 0 0        |
|  00        |
|            |
|            |
|            |
|            |
|            |
|            |
|            |
+------------+
	\end{verbatim}
	\caption{Condições do tabuleiro 10x10 após 2a iteração.}
\end{figure}
\begin{figure}[!htb]
	\begin{verbatim}
+------------+
|            |
|            |
|  0         |
|   00       |
|  00        |
|            |
|            |
|            |
|            |
|            |
|            |
|            |
+------------+
	\end{verbatim}
	\caption{Condições do tabuleiro 10x10 após 3a iteração.}
\end{figure}
\begin{figure}[!htb]
	\begin{verbatim}
+------------+
|            |
|            |
|   0        |
|    0       |
|  000       |
|            |
|            |
|            |
|            |
|            |
|            |
|            |
+------------+
	\end{verbatim}
	\caption{Condições do tabuleiro 10x10 após 4a iteração.}
	\label{fig:life1-10-end}
\end{figure}

Foram realizados também testes em tabuleiros de tamanho 1000x1000, de forma que na Figura \ref{fig:life1-1000-start} temos as 10 primeiras linhas e colunas da configuração inicial do tabuleiro e na Figura \ref{fig:life1-10-end}, as 10 últimas linhas e colunas da geração final do tabuleiro após $4(N-5)$ iterações do algoritmo.

\begin{figure}[!htb]
	\begin{verbatim}
+------------+
|            |
|  0         |
|   0        |
| 000        |
|            |
|            |
|            |
|            |
|            |
|            |
|            |
|            |
+------------+
	\end{verbatim}
	\caption{Condições iniciais do tabuleiro 10x10.}
	\label{fig:life1-1000-start}
\end{figure}


\begin{figure}[!htb]
	\begin{verbatim}
+------------+
|            |
|            |
|            |
|            |
|            |
|            |
|            |
|            |
|         0  |
|          0 |
|        000 |
|            |
+------------+

	\end{verbatim}
	\caption{Condições do tabuleiro 1000x1000 após 4(N-5) iterações.}
	\label{fig:life1-1000-end}
\end{figure}

Finalmente, na Tabela \ref{tab:exp01} temos a avaliação do desempenho do algoritmo Original e Modificado, após $4(N-3)$ iterações, e percebemos o tempo de execução foi ligeiramente maior, assim como o número de ciclos por elemento, porém, percebe-se que esse tipo de otimização não surte grande efeito, já que a variação foi menor do que um décimo de segundo.

\begin{table}[htb!]
	\centering
	\begin{tabular}{lrr}%
		\bfseries Mode & \bfseries Tempo(s)& \bfseries CPE
		\csvreader[]{tables/ex01.csv}{}
		{\\ \csvcoli & \csvcolii & \csvcoliii}

	\end{tabular}
	\caption{\label{tab:exp01}Avaliação do desempenho do algoritmo Original e Modificado.}
\end{table}


\section{Experimento II - GPROF }

O segundo experimento consiste na simples utilização do gprof para verificar os tempos de execução por rotina do algoritmo original. Como só há uma função, a função ``evolve'' tem 100\% do tempo de execução, como visto na Figura \ref{fig:ex02}:

\begin{figure}[!htb]
	\begin{verbatim}
Flat profile:

Each sample counts as 0.01 seconds.
%   cumulative   self              self     total           
time   seconds   seconds    calls  Ts/call  Ts/call  name    
100.54      7.52     7.52                             evolve

	\end{verbatim}
	\caption{\label{fig:ex02} Saída do profiler gprof, após a execução do algoritmo original do jogo da vida.}
\end{figure}

\bibliographystyle{IEEEtran}

%\bibliography{references}

\end{document}
