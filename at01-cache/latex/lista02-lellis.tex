\documentclass[conference]{IEEEtran}
\usepackage[brazilian]{babel}
\usepackage[utf8]{inputenc}
\usepackage[T1]{fontenc}
\usepackage{cite}
\ifCLASSINFOpdf
  \usepackage[pdftex]{graphicx}
  % declare the path(s) where your graphic files are
  % \graphicspath{{../pdf/}{../jpeg/}}
  % and their extensions so you won't have to specify these with
  % every instance of \includegraphics
  % \DeclareGraphicsExtensions{.pdf,.jpeg,.png}
\else
  % or other class option (dvipsone, dvipdf, if not using dvips). graphicx
  % will default to the driver specified in the system graphics.cfg if no
  % driver is specified.
  \usepackage[dvips]{graphicx}
  % declare the path(s) where your graphic files are
  % \graphicspath{{../eps/}}
  % and their extensions so you won't have to specify these with
  % every instance of \includegraphics
  % \DeclareGraphicsExtensions{.eps}
\fi
\usepackage[cmex10]{amsmath}
\usepackage{algorithmic}
\usepackage{array}
\usepackage{mdwmath}
\usepackage{mdwtab}
%\usepackage{eqparbox}
\usepackage[tight,footnotesize]{subfigure}
%\usepackage[caption=false]{caption}
\usepackage[font=footnotesize]{subfig}
%\usepackage{stfloats}
\usepackage{url}
\usepackage{csvsimple,longtable,booktabs}
\usepackage{pgfplots}
\usepackage{pgfplotstable}
\hyphenation{op-tical net-works semi-conduc-tor}

\begin{document}
\title{Programação de Alto Desempenho\\
\large Atividade 2 - Otimizando o desempenho de códigos para afinidade de memória}

\author{\IEEEauthorblockN{Lucas Santana Lellis - 69618}
\IEEEauthorblockA{PPGCC - Instituto de Ciência e Tecnologia\\
	Universidade Federal de São Paulo} }

% make the title area
\maketitle

%\IEEEpeerreviewmaketitle

\section{Introdução}
Nesta atividade foram realizados experimentos relacionados com a otimização do desempenho de algoritmos quanto à afinidade de memória.\\
Cada experimento foi realizado 5 vezes, e os resultados apresentados são a média dos resultados obtidos em cada um deles.

Todos os programas foram feitos em C, utilizando a biblioteca PAPI para estimar o tempo total de processamento, quantidade de chache misses em memória cache L2, e o total de operações de ponto flutuante.

As especificações da máquina utilizada estão disponíveis na Tabela \ref{tab:cpu}.

\begin{table}[h]
\centering
\caption{Especificações da Máquina}
\label{tab:cpu}
\begin{tabular}{ll}
 CPU & Intel Core i5 - 3470\\
 Cores & 4\\
 Threads & 4\\
 Clock & 3.2 GHz\\
 Cache L3 & 6144 KB \\
 Cache L2 & 256 KB * 4 \\
 Hardware Counters & 11 \\
 RAM & 8 Gb \\
 SO & Fedora 4
\end{tabular}
\end{table}

\section{Experimento 1}
Neste experimento foi implementado o algoritmo tradicional para multiplicação de matrizes, sem blocagem, para verificar a diferença no desempenho causada pela mudança da hierarquia dos laços: ijk, ikj, jik, jki, kij e kji.

\begin{table}[h]
	\centering
	\caption{Desempenho obtido no exp 1}
	\label{tab:exp01}
	\begin{tabular}{cccccc}%
		\bfseries Size & \bfseries Mode & \bfseries Time(ms) & \bfseries L2\_DCM & \bfseries MFLOPS & \bfseries CPI
		\csvreader[]{tables/ex01.csv}{}
		{\\\csvcoli & \csvcolii & \csvcoliii & \csvcoliv & \csvcolv & \csvcolvi}
		
	\end{tabular}
\end{table}



\begin{figure}
\centering
\begin{tikzpicture}
\begin{axis}[xlabel={Tamanho}, ylabel={Tempo (ms)}, legend pos=north west]
\addplot gnuplot [raw gnuplot] {plot 'plots/ex01-time.txt' index 0};
\addplot gnuplot [raw gnuplot] {plot 'plots/ex01-time.txt' index 1};
\addplot gnuplot [raw gnuplot] {plot 'plots/ex01-time.txt' index 2};
\addplot gnuplot [raw gnuplot] {plot 'plots/ex01-time.txt' index 3};
\addplot gnuplot [raw gnuplot] {plot 'plots/ex01-time.txt' index 4};
\addplot gnuplot [raw gnuplot] {plot 'plots/ex01-time.txt' index 5};
\legend{$IJK$, $IKJ$, $JIK$, $JKI$, $KIJ$, $KJI$ }
\end{axis}
\end{tikzpicture}
\caption{Comparação do tempo de execução entre as diferentes hierarquias.}
\label{fig:exp01-time}
\end{figure}


\begin{figure}
	\centering
	\begin{tikzpicture}
	\begin{axis}[xlabel={Tamanho}, ylabel={Tempo (ms)}, legend pos=north west]
	\addplot gnuplot [raw gnuplot] {plot 'plots/ex01-l2dcm.txt' index 0};
	\addplot gnuplot [raw gnuplot] {plot 'plots/ex01-l2dcm.txt' index 1};
	\addplot gnuplot [raw gnuplot] {plot 'plots/ex01-l2dcm.txt' index 2};
	\addplot gnuplot [raw gnuplot] {plot 'plots/ex01-l2dcm.txt' index 3};
	\addplot gnuplot [raw gnuplot] {plot 'plots/ex01-l2dcm.txt' index 4};
	\addplot gnuplot [raw gnuplot] {plot 'plots/ex01-l2dcm.txt' index 5};
	\legend{$IJK$, $IKJ$, $JIK$, $JKI$, $KIJ$, $KJI$ }
	\end{axis}
	\end{tikzpicture}
	\caption{Comparação de cache-misses em memória cache L2 entre as diferentes hierarquias.}
	\label{fig:exp01-l2dcm}
\end{figure}

\section{Experimento 1}
Neste experimento foi implementado o algoritmo tradicional para multiplicação de matrizes, sem blocagem, para verificar a diferença no desempenho causada pela mudança da hierarquia dos laços: ijk, ikj, jik, jki, kij e kji.

\begin{table}[h]
	\centering
	\caption{Desempenho obtido no exp 1}
	\label{tab:exp01}
	\begin{tabular}{cccccc}%
		\bfseries Size & \bfseries Mode & \bfseries Time(ms) & \bfseries L2\_DCM & \bfseries MFLOPS & \bfseries CPI
		\csvreader[]{tables/ex01.csv}{}
		{\\\csvcoli & \csvcolii & \csvcoliii & \csvcoliv & \csvcolv & \csvcolvi}
		
	\end{tabular}
\end{table}



\begin{figure}
	\centering
	\begin{tikzpicture}
	\begin{axis}[xlabel={Tamanho}, ylabel={Tempo (ms)}, legend pos=north west]
	\addplot gnuplot [raw gnuplot] {plot 'plots/ex01-time.txt' index 0};
	\addplot gnuplot [raw gnuplot] {plot 'plots/ex01-time.txt' index 1};
	\addplot gnuplot [raw gnuplot] {plot 'plots/ex01-time.txt' index 2};
	\addplot gnuplot [raw gnuplot] {plot 'plots/ex01-time.txt' index 3};
	\addplot gnuplot [raw gnuplot] {plot 'plots/ex01-time.txt' index 4};
	\addplot gnuplot [raw gnuplot] {plot 'plots/ex01-time.txt' index 5};
	\legend{$IJK$, $IKJ$, $JIK$, $JKI$, $KIJ$, $KJI$ }
	\end{axis}
	\end{tikzpicture}
	\caption{Comparação do tempo de execução entre as diferentes hierarquias.}
	\label{fig:exp01-time}
\end{figure}


\begin{figure}
	\centering
	\begin{tikzpicture}
	\begin{axis}[xlabel={Tamanho}, ylabel={Tempo (ms)}, legend pos=north west]
	\addplot gnuplot [raw gnuplot] {plot 'plots/ex01-l2dcm.txt' index 0};
	\addplot gnuplot [raw gnuplot] {plot 'plots/ex01-l2dcm.txt' index 1};
	\addplot gnuplot [raw gnuplot] {plot 'plots/ex01-l2dcm.txt' index 2};
	\addplot gnuplot [raw gnuplot] {plot 'plots/ex01-l2dcm.txt' index 3};
	\addplot gnuplot [raw gnuplot] {plot 'plots/ex01-l2dcm.txt' index 4};
	\addplot gnuplot [raw gnuplot] {plot 'plots/ex01-l2dcm.txt' index 5};
	\legend{$IJK$, $IKJ$, $JIK$, $JKI$, $KIJ$, $KJI$ }
	\end{axis}
	\end{tikzpicture}
	\caption{Comparação de cache-misses em memória cache L2 entre as diferentes hierarquias.}
	\label{fig:exp01-l2dcm}
\end{figure}




\bibliographystyle{IEEEtran}

% \bibliography{references}

\end{document}


