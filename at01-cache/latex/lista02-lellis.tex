\documentclass[conference]{IEEEtran}
\usepackage[brazilian]{babel}
\usepackage[utf8]{inputenc}
\usepackage[T1]{fontenc}
\usepackage{cite}
\ifCLASSINFOpdf
  \usepackage[pdftex]{graphicx}
  % declare the path(s) where your graphic files are
  % \graphicspath{{../pdf/}{../jpeg/}}
  % and their extensions so you won't have to specify these with
  % every instance of \includegraphics
  % \DeclareGraphicsExtensions{.pdf,.jpeg,.png}
\else
  % or other class option (dvipsone, dvipdf, if not using dvips). graphicx
  % will default to the driver specified in the system graphics.cfg if no
  % driver is specified.
  \usepackage[dvips]{graphicx}
  % declare the path(s) where your graphic files are
  % \graphicspath{{../eps/}}
  % and their extensions so you won't have to specify these with
  % every instance of \includegraphics
  % \DeclareGraphicsExtensions{.eps}
\fi
\usepackage[cmex10]{amsmath}
\usepackage{algorithmic}
\usepackage{array}
\usepackage{mdwmath}
\usepackage{mdwtab}
%\usepackage{eqparbox}
\usepackage[tight,footnotesize]{subfigure}
%\usepackage[caption=false]{caption}
\usepackage[font=footnotesize]{subfig}
%\usepackage{stfloats}
\usepackage{url}
\usepackage{csvsimple,longtable,booktabs}
\usepackage{pgfplots}
\usepackage{pgfplotstable}
\hyphenation{op-tical net-works semi-conduc-tor}

\begin{document}
\title{Programação de Alto Desempenho\\
\large Atividade 2 - Otimizando o desempenho de códigos para afinidade de memória}

\author{\IEEEauthorblockN{Lucas Santana Lellis - 69618}
\IEEEauthorblockA{PPGCC - Instituto de Ciência e Tecnologia\\
	Universidade Federal de São Paulo} }

% make the title area
\maketitle

%\IEEEpeerreviewmaketitle

\section{Introdução}
Nesta atividade foram realizados experimentos relacionados com a otimização do desempenho de algoritmos quanto à afinidade de memória.\\
Cada experimento foi realizado 5 vezes, e os resultados apresentados são a média dos resultados obtidos em cada um deles.

Todos os programas foram feitos em C, utilizando a biblioteca PAPI para estimar o tempo total de processamento, quantidade de chache misses em memória cache L2, e o total de operações de ponto flutuante.

As especificações da máquina utilizada estão disponíveis na Tabela \ref{tab:cpu}.

\begin{table}[htb!]
\centering
\caption{Especificações da Máquina}
\label{tab:cpu}
\begin{tabular}{ll}
 CPU & Intel Core i5 - 3470\\
 Cores & 4\\
 Threads & 4\\
 Clock & 3.2 GHz\\
 Cache L3 & 6144 KB \\
 Cache L2 & 256 KB * 4 \\
 Hardware Counters & 11 \\
 RAM & 8 Gb \\
 SO & Fedora 4
\end{tabular}
\end{table}

\section{Experimento 1}
Nesse experimento foi implementado o algoritmo tradicional para multiplicação de matrizes, sem blocagem, para verificar a diferença no desempenho causada pela mudança da hierarquia dos laços: ijk, ikj, jik, jki, kij e kji.

\begin{table}[htb!]
	\centering
	\caption{Desempenho obtido no exp 1}
	\label{tab:exp01}
	\begin{tabular}{cccccc}%
		\bfseries Size & \bfseries Mode & \bfseries Time(ms) & \bfseries L2\_DCM & \bfseries MFLOPS & \bfseries CPI
		\csvreader[]{tables/ex01.csv}{}
		{\\\csvcoli & \csvcolii & \csvcoliii & \csvcoliv & \csvcolv & \csvcolvi}

	\end{tabular}
\end{table}

%
%
% \begin{figure}[htb!]
% \centering
% \begin{tikzpicture}
% \begin{axis}[xlabel={Tamanho}, ylabel={Tempo (ms)}, legend pos=north west]
% \addplot gnuplot [raw gnuplot] {plot 'plots/ex01-time.txt' index 0};
% \addplot gnuplot [raw gnuplot] {plot 'plots/ex01-time.txt' index 1};
% \addplot gnuplot [raw gnuplot] {plot 'plots/ex01-time.txt' index 2};
% \addplot gnuplot [raw gnuplot] {plot 'plots/ex01-time.txt' index 3};
% \addplot gnuplot [raw gnuplot] {plot 'plots/ex01-time.txt' index 4};
% \addplot gnuplot [raw gnuplot] {plot 'plots/ex01-time.txt' index 5};
% \legend{$IJK$, $IKJ$, $JIK$, $JKI$, $KIJ$, $KJI$ }
% \end{axis}
% \end{tikzpicture}
% \caption{Comparação do tempo de execução entre as diferentes hierarquias.}
% \label{fig:exp01-time}
% \end{figure}
%
%
% \begin{figure}[htb!]
% 	\centering
% 	\begin{tikzpicture}
% 	\begin{axis}[xlabel={Tamanho}, ylabel={Cache misses}, legend pos=north west]
% 	\addplot gnuplot [raw gnuplot] {plot 'plots/ex01-l2dcm.txt' index 0};
% 	\addplot gnuplot [raw gnuplot] {plot 'plots/ex01-l2dcm.txt' index 1};
% 	\addplot gnuplot [raw gnuplot] {plot 'plots/ex01-l2dcm.txt' index 2};
% 	\addplot gnuplot [raw gnuplot] {plot 'plots/ex01-l2dcm.txt' index 3};
% 	\addplot gnuplot [raw gnuplot] {plot 'plots/ex01-l2dcm.txt' index 4};
% 	\addplot gnuplot [raw gnuplot] {plot 'plots/ex01-l2dcm.txt' index 5};
% 	\legend{$IJK$, $IKJ$, $JIK$, $JKI$, $KIJ$, $KJI$ }
% 	\end{axis}
% 	\end{tikzpicture}
% 	\caption{Comparação de cache-misses em memória cache L2 entre as diferentes hierarquias.}
% 	\label{fig:exp01-l2dcm}
% \end{figure}

\section{Experimento 2}
Nesse experimento foi implementado o algoritmo para multiplicação de matrizes com blocagem, para verificar a diferença no desempenho causada pela mudança do tamanho do bloco para 2, 4, 16 e 64.


\begin{table}[htb!]
	\centering
	\caption{Desempenho obtido no exp 2}
	\label{tab:exp02}
	\begin{tabular}{cccccc}%
		\bfseries Size & \bfseries Block & \bfseries Time(ms) & \bfseries L2\_DCM & \bfseries MFLOPS & \bfseries CPI
		\csvreader[]{tables/ex02.csv}{}
		{\\\csvcoli & \csvcolii & \csvcoliii & \csvcoliv & \csvcolv & \csvcolvi}

	\end{tabular}
\end{table}

%
%
% \begin{figure}[htb!]
% 	\centering
% 	\begin{tikzpicture}
% 	\begin{axis}[xlabel={Tamanho}, ylabel={Tempo (ms)}, legend pos=north west]
% 	\addplot gnuplot [raw gnuplot] {plot 'plots/ex02-time.txt' index 0};
% 	\addplot gnuplot [raw gnuplot] {plot 'plots/ex02-time.txt' index 1};
% 	\addplot gnuplot [raw gnuplot] {plot 'plots/ex02-time.txt' index 2};
% 	\addplot gnuplot [raw gnuplot] {plot 'plots/ex02-time.txt' index 3};
% 	\legend{$2$, $4$, $16$, $64$ }
% 	\end{axis}
% 	\end{tikzpicture}
% 	\caption{Comparação do tempo de execução entre os diferentes tamanhos de blocos.}
% 	\label{fig:exp02-time}
% \end{figure}
%
%
% \begin{figure}[htb!]
% 	\centering
% 	\begin{tikzpicture}
% 	\begin{axis}[xlabel={Tamanho}, ylabel={Cache misses}, legend pos=north west]
% 	\addplot gnuplot [raw gnuplot] {plot 'plots/ex02-l2dcm.txt' index 0};
% 	\addplot gnuplot [raw gnuplot] {plot 'plots/ex02-l2dcm.txt' index 1};
% 	\addplot gnuplot [raw gnuplot] {plot 'plots/ex02-l2dcm.txt' index 2};
% 	\addplot gnuplot [raw gnuplot] {plot 'plots/ex02-l2dcm.txt' index 3};
% 	\legend{$2$, $4$, $16$, $64$ }
% 	\end{axis}
% 	\end{tikzpicture}
% 	\caption{Comparação de cache-misses em memória cache L2 entre os diferentes tamanhos de blocos.}
% 	\label{fig:exp02-l2dcm}
% \end{figure}


\section{Experimento 3}
Nesse experimento foi implementado o algoritmo de Strassen .....................


\begin{table}[htb!]
	\centering
	\caption{Desempenho obtido no exp 3}
	\label{tab:exp03}
	\begin{tabular}{cccccc}%
		\bfseries Size & \bfseries Block & \bfseries Time(ms) & \bfseries L2\_DCM & \bfseries MFLOPS & \bfseries CPI
		\csvreader[]{tables/ex03.csv}{}
		{\\\csvcoli & \csvcolii & \csvcoliii & \csvcoliv & \csvcolv & \csvcolvi}

	\end{tabular}
\end{table}

%
%
% \begin{figure}[htb!]
% 	\centering
% 	\begin{tikzpicture}
% 	\begin{axis}[xlabel={Tamanho}, ylabel={Tempo (ms)}, legend pos=north west]
% 	\addplot gnuplot [raw gnuplot] {plot 'plots/ex03-time.txt' index 0};
% 	\addplot gnuplot [raw gnuplot] {plot 'plots/ex03-time.txt' index 1};
% 	\addplot gnuplot [raw gnuplot] {plot 'plots/ex03-time.txt' index 2};
% 	\legend{$16$, $32$ , $64$ }
% 	\end{axis}
% 	\end{tikzpicture}
% 	\caption{Comparação do tempo de execução entre os diferentes tamanhos minimos.}
% 	\label{fig:exp03-time}
% \end{figure}
%
%
% \begin{figure}[htb!]
% 	\centering
% 	\begin{tikzpicture}
% 	\begin{axis}[xlabel={Tamanho}, ylabel={Cache misses}, legend pos=north west]
% 	\addplot gnuplot [raw gnuplot] {plot 'plots/ex03-l2dcm.txt' index 0};
% 	\addplot gnuplot [raw gnuplot] {plot 'plots/ex03-l2dcm.txt' index 1};
% 	\addplot gnuplot [raw gnuplot] {plot 'plots/ex03-l2dcm.txt' index 2};
% 	\legend{$16$, $32$ , $64$ }
% 	\end{axis}
% 	\end{tikzpicture}
% 	\caption{Comparação de cache-misses em memória cache L2 entre os diferentes tamanhos minimos.}
% 	\label{fig:exp03-l2dcm}
% \end{figure}


\section{Experimento 4}
Nesse experimento foi utilizada a função ..... do BLAS

\begin{table}[htb!]
	\centering
	\caption{Desempenho obtido no exp 4}
	\label{tab:exp04}
	\begin{tabular}{ccccc}%
		\bfseries Size & \bfseries Time(ms) & \bfseries L2\_DCM & \bfseries MFLOPS & \bfseries CPI
		\csvreader[]{tables/ex04.csv}{}
		{\\\csvcoli & \csvcolii & \csvcoliii & \csvcoliv & \csvcolv}

	\end{tabular}
\end{table}

Comparando os melhores resultados dos 4 experimentos (considerando os testes com matrizes de tamanho 1024x1024), fazemos então a comparação da Figura \ref{fig:compexperimentos}.

\begin{figure}[htb!]
	\centering
	\begin{tikzpicture}
	\begin{axis}[xlabel={Tamanho}, ylabel={Tempo (ms)}, legend pos=north west]
	\addplot gnuplot [raw gnuplot] {plot 'plots/compexperimentos-time.txt' index 0};
	\addplot gnuplot [raw gnuplot] {plot 'plots/compexperimentos-time.txt' index 1};
	\addplot gnuplot [raw gnuplot] {plot 'plots/compexperimentos-time.txt' index 2};
	\addplot gnuplot [raw gnuplot] {plot 'plots/compexperimentos-time.txt' index 3};
	\legend{$Ex1-IJK$, $Ex2-16 $ , $Ex3-64$, $Ex4$ }
	\end{axis}
	\end{tikzpicture}
	\caption{Comparação entre os melhores resultados dos quatro experimentos com relação ao tempo.}
	\label{fig:compexperimentos}
\end{figure}


\section{Experimento 5}
Nesse experimento foi feita a comparação do desempenho da técnica de fusão de laços trabalhando com diferentes estruturas de dados.

\begin{table}[htb!]
	\centering
	\caption{Desempenho obtido no exp 5}
	\label{tab:exp02}
	\begin{tabular}{cccccc}%
		\bfseries Size & \bfseries Mode & \bfseries Time(ms) & \bfseries L2\_DCM & \bfseries MFLOPS & \bfseries CPI
		\csvreader[]{tables/ex05.csv}{}
		{\\\csvcoli & \csvcolii & \csvcoliii & \csvcoliv & \csvcolv & \csvcolvi}

	\end{tabular}
\end{table}


\section{Experimento 6}
Nesse experimento foi feita a comparação do desempenho da técnica de fusão de laços.

\begin{table}[htb!]
	\centering
	\caption{Desempenho obtido no exp 5}
	\label{tab:exp02}
	\begin{tabular}{cccccc}%
		\bfseries Size & \bfseries Mode & \bfseries Time(ms) & \bfseries L2\_DCM & \bfseries MFLOPS & \bfseries CPI
		\csvreader[]{tables/ex06.csv}{}
		{\\\csvcoli & \csvcolii & \csvcoliii & \csvcoliv & \csvcolv & \csvcolvi}

	\end{tabular}
\end{table}


% \section{Experimento XXXX}
% Nesse experimento foi utilizada a função ..... do BLAS
%
% \begin{table}[htb!]
% 	\centering
% 	\caption{Desempenho obtido no exp XXXX}
% 	\label{tab:exp02}
% 	\begin{tabular}{ccccc}%
% 		\bfseries Size & \bfseries Time(ms) & \bfseries L2\_DCM & \bfseries MFLOPS & \bfseries CPI
% 		\csvreader[]{tables/ex0XXXX.csv}{}
% 		{\\\csvcoli & \csvcolii & \csvcoliii & \csvcoliv & \csvcolv & \csvcolvi}
%
% 	\end{tabular}
% \end{table}
%
%
%
% \begin{figure}[htb!]
% 	\centering
% 	\begin{tikzpicture}
% 	\begin{axis}[xlabel={Tamanho}, ylabel={Tempo (ms)}, legend pos=north west]
% 	\addplot gnuplot [raw gnuplot] {plot 'plots/ex0XXXX-time.txt' index 0};
% 	\addplot gnuplot [raw gnuplot] {plot 'plots/ex0XXXX-time.txt' index 1};
% 	\addplot gnuplot [raw gnuplot] {plot 'plots/ex0XXXX-time.txt' index 2};
% 	\legend{$16$, $32$ , $64$ }
% 	\end{axis}
% 	\end{tikzpicture}
% 	\caption{Comparação do tempo de execução entre os diferentes tamanhos minimos.}
% 	\label{fig:exp0XXXX-time}
% \end{figure}
%
%
% \begin{figure}[htb!]
% 	\centering
% 	\begin{tikzpicture}
% 	\begin{axis}[xlabel={Tamanho}, ylabel={Cache misses}, legend pos=north west]
% 	\addplot gnuplot [raw gnuplot] {plot 'plots/ex0XXXX-l2dcm.txt' index 0};
% 	\addplot gnuplot [raw gnuplot] {plot 'plots/ex0XXXX-l2dcm.txt' index 1};
% 	\addplot gnuplot [raw gnuplot] {plot 'plots/ex0XXXX-l2dcm.txt' index 2};
% 	\legend{$16$, $32$ , $64$ }
% 	\end{axis}
% 	\end{tikzpicture}
% 	\caption{Comparação de cache-misses em memória cache L2 entre os diferentes tamanhos minimos.}
% 	\label{fig:exp0XXXX-l2dcm}
% \end{figure}



\bibliographystyle{IEEEtran}

% \bibliography{references}

\end{document}
