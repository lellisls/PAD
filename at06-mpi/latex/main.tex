\documentclass[conference]{IEEEtran}
\usepackage[brazilian]{babel}
\usepackage[utf8]{inputenc}
\usepackage[T1]{fontenc}
\usepackage{cite}
\ifCLASSINFOpdf
  \usepackage[pdftex]{graphicx}
  % declare the path(s) where your graphic files are
  % \graphicspath{{../pdf/}{../jpeg/}}
  % and their extensions so you won't have to specify these with
  % every instance of \includegraphics
  % \DeclareGraphicsExtensions{.pdf,.jpeg,.png}
\else
  % or other class option (dvipsone, dvipdf, if not using dvips). graphicx
  % will default to the driver specified in the system graphics.cfg if no
  % driver is specified.
  \usepackage[dvips]{graphicx}
  % declare the path(s) where your graphic files are
  % \graphicspath{{../eps/}}
  % and their extensions so you won't have to specify these with
  % every instance of \includegraphics
  % \DeclareGraphicsExtensions{.eps}
\fi

\usepackage{adjustbox}
\usepackage{fancyvrb}

\usepackage[cmex10]{amsmath}
\usepackage{algorithmic}
\usepackage{array}
\usepackage{mdwmath}
\usepackage{mdwtab}
%\usepackage{eqparbox}
\usepackage[tight,footnotesize]{subfigure}
%\usepackage[caption=false]{caption}
\usepackage[font=footnotesize]{subfig}
%\usepackage{stfloats}
\usepackage{url}
\usepackage{csvsimple,longtable,booktabs}
\usepackage{pgfplots}
\usepackage{pgfplotstable}
\hyphenation{op-tical net-works semi-conduc-tor}
%\usepackage{listings}

%\lstset{language=C++,
%	basicstyle=\ttfamily,
%	keywordstyle=\color{blue}\ttfamily,
%	stringstyle=\color{red}\ttfamily,
%	commentstyle=\color{green}\ttfamily,
%	morecomment=[l][\color{magenta}]{\#}
%}

\begin{document}
\title{Programação de Alto Desempenho\\
\large Atividade 6 - Exercícios em MPI}

\author{\IEEEauthorblockN{Lucas Santana Lellis - 69618}
\IEEEauthorblockA{PPGCC - Instituto de Ciência e Tecnologia\\
	Universidade Federal de São Paulo} }

% make the title area
\maketitle

%\IEEEpeerreviewmaketitle

\section{Introdução}
Nesta atividade foram realizados experimentos relacionados com a implementação de técnicas de programação paralela em sistemas de memória compartilhada utilizando OpenMP.
Cada experimento foi realizado 5 vezes, e os resultados apresentados são a média dos resultados obtidos em cada um deles, sendo calculado o speedup pela fórmula $$speedup(P) = {{\textrm{Tempo para 1 thread} }\over{\textrm{Tempo para P threads} }}$$ e a eficiência pela fórmula $$eficiencia(P) = {{speedup(P)}\over{P}}$$

Todos os programas foram feitos em C, com otimização -O3, utilizando a biblioteca PAPI para estimar o tempo total de processamento, e o número de cache misses.


As especificações da máquina utilizada estão disponíveis na Tabela \ref{tab:cpu}.

\begin{table}[htb!]
\centering
\caption{Especificações da Máquina}
\label{tab:cpu}
\begin{tabular}{lr}
CPU & Intel i7 990X \\
Cores & 6\\
Threads & 12\\
Clock & 3.47 GHz\\
Cache& 12 MB Smartcache \\
RAM & 20 GB \\
Hardware Counters & 7 \\
SO & Ubuntu 14.04 \\
Kernel & 3.13.0 \\
GCC &  6.2.0\\
\end{tabular}
\end{table}

\section{Experimento I - Multiplicação de Matrizes}


Neste experimento foi feita a implementação da multiplicação de matrizes com vetorização e blocagem para utilização efetiva de cache, utilizando OpenMP.
Foi utilizada a vetorização automática do compilador, como confirmado utilizando a flag "info-vec-all" do compilador tendo como resultado a saída disponível na Figura \ref{fig:loop-vectorized}.


\subsection{Avaliando resultado da multiplicação}
Utilizando o software R, foi possível conferir que a multiplicação das matrizes de tamanho 8x8 e com blocagem de tamanhos 2x2, 4x4 e 8x8, foi realizada com sucesso para até 4 threads.

\begin{figure}[htb!]
\begin{adjustbox}{max width=\linewidth}
	\begin{BVerbatim}
gcc-6 src/ex01.c obj/* -o bin/ex01
-Iinc -lpapi -march=native -fopenmp-ffast-math
-ftree-vectorize -O3 -fopt-info-vec-all 2>&1 |
 grep -e "LOOP VECTORIZED" -e "error" || true
 
src/ex01.c:64:13: note: LOOP VECTORIZED
	\end{BVerbatim}
\end{adjustbox}
\caption{Saída do compilador gcc, apontando vetorização automática do laço.\label{fig:loop-vectorized}}
\end{figure}

Nas Figuras \ref{fig:matA} e \ref{fig:matB} estão as matrizes de entrada, foi feita a multiplicação dessas matrizes em um outro software confiável (R), e o resultado está na Figura \ref{fig:matC-R}, enfim, para todos os testes executados, obtivemos a mesma saída, representada na Figura \ref{fig:matC}.f Assim, comparando os resultados é fácil observar que a multiplicação está sendo feita corretamente.

\begin{figure}[htb!]
\begin{adjustbox}{max width=\linewidth}
	\begin{BVerbatim}
77.26 54.95 97.36 30.08 77.49 52.89 61.91 09.30
93.01 98.07 26.35 34.81 01.79 02.58 16.81 02.21
55.81 06.73 33.68 29.09 51.98 84.37 58.02 92.30
39.49 46.66 96.38 25.62 37.06 43.71 30.64 11.47
30.14 03.15 18.54 38.51 77.74 11.46 43.59 77.36
94.13 75.96 32.54 15.72 29.80 50.51 68.89 10.42
07.24 26.37 09.34 86.87 51.28 13.41 02.09 41.48
78.32 77.30 82.55 07.50 15.98 99.84 47.04 10.94
	\end{BVerbatim}
\end{adjustbox}
\caption{Matriz de entrada A.\label{fig:matA}}
\end{figure}


\begin{figure}[htb!]
	\begin{adjustbox}{max width=\linewidth}
		\begin{BVerbatim}
53.38 49.13 12.99 75.91 51.64 73.20 12.52 14.88
18.77 32.11 42.48 32.74 92.06 82.02 02.43 94.08
20.69 53.17 36.81 11.17 45.50 13.89 57.19 14.55
90.37 29.36 44.99 61.69 28.21 17.39 45.93 86.46
81.73 63.82 32.24 34.04 22.88 99.28 03.76 83.08
24.02 54.03 01.59 33.12 59.44 23.96 25.51 99.03
13.58 89.30 64.88 50.61 98.33 94.87 90.77 59.74
80.96 10.86 93.02 75.87 58.01 48.39 10.26 54.28
		\end{BVerbatim}
	\end{adjustbox}
	\caption{Matriz de entrada B.\label{fig:matB}}
\end{figure}

\begin{figure}[htb!]
\begin{adjustbox}{max width=\linewidth}
\begin{BVerbatim}
> mat <- scan('tests/matA.txt')
Read 64 items
> matA <- matrix(mat, ncol=8, byrow=TRUE)
> mat <- scan('tests/matB.txt')
Read 64 items
> matB <- matrix(mat, ncol=8, byrow=TRUE)
> matA %*% matB
[,1]      [,2]      [,3]     [,4]     [,5]     [,6]      [,7]     [,8]
[1,] 19085.60 25052.627 15739.183 18835.35 25870.69 27321.73 15405.991 26215.50
[2,] 11112.08 11920.428  9268.277 13877.77 17987.71 17764.61  6129.639 15531.88
[3,] 20966.57 19662.331 17719.448 21130.57 23118.34 22763.81 12538.554 25618.59
[4,] 12716.70 16902.660 11514.659 12312.51 18567.07 16690.43 11449.760 18455.39
[5,] 19015.72 14011.617 15489.058 16075.05 15010.09 19263.26  8549.099 18740.86
[6,] 13972.23 20151.302 12834.484 17929.95 24840.34 25054.49 11706.685 22538.45
[7,] 16824.95  8883.894 11134.989 12318.85 10259.21 11951.11  5829.013 18200.02
[8,] 13246.08 21673.026 12420.705 16922.20 26688.47 22320.94 13222.982 24906.10
\end{BVerbatim}
\end{adjustbox}
\caption{Matriz de saída C - obtida no R. \label{fig:matC-R}}
\end{figure}

\begin{figure}[htb!]
	\begin{adjustbox}{max width=\linewidth}
		\begin{BVerbatim}
19085.60 25052.62 15739.18 18835.35 25870.69 27321.73 15405.99 26215.50
11112.08 11920.43 09268.28 13877.76 17987.71 17764.61 06129.64 15531.88
20966.57 19662.33 17719.45 21130.57 23118.35 22763.81 12538.55 25618.59
12716.70 16902.66 11514.66 12312.51 18567.07 16690.43 11449.76 18455.39
19015.71 14011.62 15489.06 16075.05 15010.09 19263.26 08549.10 18740.86
13972.23 20151.30 12834.48 17929.95 24840.34 25054.49 11706.68 22538.45
16824.95 08883.89 11134.99 12318.85 10259.21 11951.11 05829.01 18200.02
13246.08 21673.03 12420.70 16922.20 26688.47 22320.95 13222.98 24906.10
		\end{BVerbatim}
	\end{adjustbox}
	\caption{Matriz de saída C - obtida em todos os testes. \label{fig:matC}}
\end{figure}

\subsection{Comparando speedup e eficiência}
Na Tabela \ref{tab:ex01} estão disponíveis os resultados dos testes, para 1, 2, 4, 8 e 12 threads, da multiplicação de matrizes com blocagem e vetorização automática, utilizando blocos de tamanho 128x128. Na Figura \ref{fig:ex01-speedup}, está disponível a comparação do speedup da multiplicação de matrizes de largura 1024 e 4096.

\begin{table}[htb!]
	\begin{adjustbox}{max width=\linewidth}
	\centering
	\begin{tabular}{lllrrrr}%
		\bfseries Tam. & \bfseries Threads & \bfseries Tempo(s) & \bfseries Cache Miss & \bfseries Speedup & \bfseries Eficiencia
		\csvreader[]{tables/ex01.csv}{}
		{\\ \csvcoli & \csvcoliii & \csvcoliv & \csvcolv & \csvcolvi & \csvcolvii}
	\end{tabular}
	\end{adjustbox}
	\caption{\label{tab:ex01}Avaliação do desempenho do algoritmo de multiplicação com blocagem e vetorização com 1, 2, 4, 8 e 12 threads.}
\end{table}

\begin{figure}[htb!]
	\centering
	\begin{tikzpicture}
	\begin{axis}[xlabel={Threads}, ylabel={Speedup}, legend pos=north west]
	\addplot gnuplot [raw gnuplot] {plot 'plots/ex01.txt' index 0};
	\addplot gnuplot [raw gnuplot] {plot 'plots/ex01.txt' index 1};
	\legend{$1024$, $4096$}
	\end{axis}
	\end{tikzpicture}
	\caption{Comparação do speedup da multiplicação de matrizes para 1, 2, 4, 8 e 12 Threads. \label{fig:ex01-speedup}}
\end{figure}

\section{Experimento II - Odd-Even Sort}
Nesse experimento foi feita a paralelização do algoritmo Odd-Even Sort, realizando a ordenação de valores pseudo aleatórios com 1, 2 e 4 threads obteve-se os mesmos resultados, como visto na Figura \ref{fig:ex02}.
\begin{figure}[htb!]
	\begin{adjustbox}{max width=\linewidth}
		\begin{BVerbatim}
Entrada  :   6 8 5 3 6 9 8 1 9 4
1 Thread :   1 3 4 5 6 6 8 8 9 9
2 Threads:   1 3 4 5 6 6 8 8 9 9
4 Threads:   1 3 4 5 6 6 8 8 9 9
8 Threads:   1 3 4 5 6 6 8 8 9 9
12 Threads:  1 3 4 5 6 6 8 8 9 9
		\end{BVerbatim}
	\end{adjustbox}
	\caption{Resultados dos testes do algoritmo Odd-Even sort de uma lista de 10 elementos para 1, 2, 4, 8 e 12 threads.\label{fig:ex02}}
\end{figure}

Foram então realizados testes para vetores de tamanho $4*10^4$, os resultados estão disponíveis na Tabela \ref{tab:ex02}, e o gráfico de speedup está na Figura \ref{fig:ex02-speedup}.

\begin{table}[htb!]
	\begin{adjustbox}{max width=\linewidth}
		\centering
		\begin{tabular}{lllrrrr}%
			\bfseries Tam. & \bfseries Threads & \bfseries Tempo(s) & \bfseries Cache Miss & \bfseries Speedup & \bfseries Eficiencia
			\csvreader[]{tables/ex02.csv}{}
			{\\ \csvcoli & \csvcolii & \csvcoliii & \csvcoliv & \csvcolv & \csvcolvi}
		\end{tabular}
	\end{adjustbox}
	\caption{\label{tab:ex02}Avaliação do desempenho do algoritmo Odd-Even sort para 1, 2, 4, 8 e 12 threads.}
\end{table}

\begin{figure}[htb!]
	\centering
	\begin{tikzpicture}
	\begin{axis}[xlabel={Threads}, ylabel={Speedup}, legend pos=north west]
	\addplot gnuplot [raw gnuplot] {plot 'plots/ex02.txt' index 0};
	\end{axis}
	\end{tikzpicture}
	\caption{Comparação do speedup do Odd-Even sort para 1, 2, 4, 8 e 12 Threads. 
	\label{fig:ex02-speedup}}
\end{figure}



\section{Experimento III - Contagem de números}
Nesse experimento foi implementada a paralelização de um algoritmo para contagem da ocorrência de números em um vetor, para 1, 2 e 4 threads foi possível identificar que o valor total da soma de ocorrências de cada número é igual a $10^8$, que é igual ao tamanho do vetor de entrada, o que confirma a validade da solução utilizada para paralelização.

A comparação de tempo de execução, speedup e eficiência para 1, 2 e 4 threads está disponível na Tabela \ref{tab:ex03}.

\begin{table}[htb!]
	\begin{adjustbox}{max width=\linewidth}
		\centering
		\begin{tabular}{lllrrrr}%
			\bfseries Tam. & \bfseries Threads & \bfseries Tempo(s) & \bfseries Cache Miss & \bfseries Speedup & \bfseries Eficiencia
			\csvreader[]{tables/ex03.csv}{}
			{\\ \csvcoli & \csvcolii & \csvcoliii & \csvcoliv & \csvcolv & \csvcolvi}
		\end{tabular}
	\end{adjustbox}
	\caption{\label{tab:ex03}Avaliação do desempenho do algoritmo de contagem de números para 1, 2, 4, 8 e 12 threads.}
\end{table}

\begin{figure}[htb!]
	\centering

	\begin{tikzpicture}
	\begin{axis}[xlabel={Threads}, ylabel={Speedup}, legend pos=north west]
	\addplot gnuplot [raw gnuplot] {plot 'plots/ex03.txt' index 0};
	\end{axis}
	\end{tikzpicture}
	\caption{Comparação do speedup do algoritmo de contagem para 1, 2, 4, 8 e 12 Threads. 
	\label{fig:ex03-speedup}}
\end{figure}

\section{Experimento IV - Conjectura de Goldbach}
Nesse experimento foi feita a comparação entre diferentes modos de escalonamento para a paralelização do algoritmo da conjectura de goldbach.


A comparação de tempo de execução, speedup e eficiência para 1, 2 e 4 threads usando os diferentes tipos de escalonamento está disponível na Tabela \ref{tab:ex04}.

\begin{table}[htb!]
	\begin{adjustbox}{max width=\linewidth}
		\centering
		\begin{tabular}{lllrrrr}%
			\bfseries Escalonamento & \bfseries Threads & \bfseries Tempo(s) & \bfseries Cache Miss & \bfseries Speedup & \bfseries Eficiencia
			\csvreader[]{tables/ex04.csv}{}
			{\\ \csvcoli & \csvcolii & \csvcoliii & \csvcoliv & \csvcolv & \csvcolvi}
		\end{tabular}
	\end{adjustbox}
	\caption{\label{tab:ex04}Avaliação do desempenho do algoritmo de contagem de números para 1, 2, 4, 8 e 12 threads.}
\end{table}

\begin{figure}[htb!]
	\centering
	
	\begin{tikzpicture}
	\begin{axis}[xlabel={Threads}, ylabel={Speedup}, legend pos=north west]
	\addplot gnuplot [raw gnuplot] {plot 'plots/ex04.txt' index 0};
	\addplot gnuplot [raw gnuplot] {plot 'plots/ex04.txt' index 1};
	\addplot gnuplot [raw gnuplot] {plot 'plots/ex04.txt' index 2};
	\addplot gnuplot [raw gnuplot] {plot 'plots/ex04.txt' index 3};
	\addplot gnuplot [raw gnuplot] {plot 'plots/ex04.txt' index 4};
	\legend{static 10, static 5, static 2, dynamic, guided};
	\end{axis}
	\end{tikzpicture}
	\caption{Comparação do speedup do algoritmo de goldbach, para diferentes tipos de escalonamento e para 1, 2, 4, 8 e 12 Threads. 
		\label{fig:ex04-speedup}}
\end{figure}


\section{Experimento V - Jogo da Vida}
Nesse experimento foi implementada a paralelização de um algoritmo do jogo da vida.
Para 1, 2 e 4 threads foi possível identificar que a solução permanece a mesma, em um tabuleiro de tamanho 1000x1000.

\begin{table}[htb!]
	\begin{adjustbox}{max width=\linewidth}
		\centering
		\begin{tabular}{lllrrrr}%
			\bfseries Tamanho & \bfseries Threads & \bfseries Tempo(s) & \bfseries Cache Miss & \bfseries Speedup & \bfseries Eficiencia
			\csvreader[]{tables/ex05.csv}{}
			{\\ \csvcoli & \csvcolii & \csvcoliii & \csvcoliv & \csvcolv & \csvcolvi}
		\end{tabular}
	\end{adjustbox}
	\caption{\label{tab:ex05}Avaliação do desempenho do algoritmo de contagem de números para 1, 2, 4, 8 e 12 threads.}
\end{table}

\begin{figure}[htb!]
	\centering
	
	\begin{tikzpicture}
	\begin{axis}[xlabel={Threads}, ylabel={Speedup}, legend pos=north west]
	\addplot gnuplot [raw gnuplot] {plot 'plots/ex05.txt' index 0};
	\end{axis}
	\end{tikzpicture}
	\caption{Comparação do speedup do algoritmo do jogo da vida para 1, 2, 4, 8 e 12 Threads. 
		\label{fig:ex05-speedup}}
\end{figure}

\bibliographystyle{IEEEtran}

%\bibliography{references}

\end{document}
